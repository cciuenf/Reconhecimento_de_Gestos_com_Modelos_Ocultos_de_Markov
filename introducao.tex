\chapter{Introdução}

\section{Contextualização e Motivação}

    O entretenimento digital vem sendo cada vez mais popular entre as crianças e os adolescentes e atualmente é tão abrangente que engloba um público mais adulto e maduro
e um outro grupo de jogadores casuais que antigamente não jogavam por geralmente encontrar dificuldades em manipular o joystick usual,que agora jogam videogames como
o Nintendo Wii, o PS Move no Playstation 3 e o Kinect no Xbox360. Muitos fabricantes de jogos eletrônicos, se empenham em criar efeitos mais atrativos para os
usuários, uns recorrem a gráficos e animações mais realistas enquanto outros se preocupam em criar  uma experiência de jogabilidade inovadora.\cite{WiiWillRockYou}

Nos jogos fisicamente interativos\footnote{\textit{What is motion-controlled gaming? Definition from WhatIs.com.}
http://whatis.techtarget.com/definition/motion-gaming.html. Acesso  em: 14/09/2011.}, o usuário movimenta partes do corpo para atuar direta ou indiretamente com
o jogo. Esses movimentos são geralmente capturados por controles com sensores de movimento na maioria das vezes com uma câmera, que através de processamento de
imagens, serão traduzidos e retornados como comandos no ambiente virtual do jogo. Essa interação pode ser através de um objeto animado que representa o jogador,
chamado de avatar realizando ações que o usuário realiza na área de visão da câmera.

Atualmente a indústria de jogos tem um movimento financeiro comparável a indústria cinematográfica e um retorno lucrativo muitas vezes maior.
Ela cresceu muito e enquanto antigamente era comum apenas um programador desenvolver um jogo, hoje é necessário quase sempre, centenas de pessoas e milhares de
dólares e alguns anos para a conclusão de um projeto. No fim do ano passado \textit{Call of Duty: Black Ops}\footnote{\textit{Call of Duty: Black Ops} é um jogo de
tiro em primeira pessoa desenvolvido pela Treyarch, editado pela Activision e lançado mundialmente em 9 de novembro de 2010 para as plataformas Microsoft Windows,
 Xbox 360, Playstation 3, Wii e Nintendo DS.} foi o maior lançamento da história do entretenimento,
vendendo no dia do lançamento 5,6 milhões de cópias nos Estados Unidos e Reino Unido, já vendeu mais de 13 milhões e é o título mais vendido da história no mercado
americano. Apenas para efeito de comparação, o filme \textit{Avatar}\footnote{\textit{Avatar} é um filme americano de ficção científica de 2009, escrito e dirigido
por James Cameron, que está em primeiro lugar no ranking de bilheteria, que rendeu US\$ 2,78 bilhões.} vendeu 3,2 milhões de DVDs e Blu-Rays mundialmente no dia do
 lançamento.\cite{TamanhoIndustriaGames}

 Entretenimento por computador ou videogames fazem um papel importante na vida de muitas pessoas e os pais acreditam que existe um impacto positivo em jogar por videogames.
De acordo com a Associação de Software de Entretenimentos do Estados Unidos\cite{ESA}, 72\% dos chefes de família jogam por computador ou vídeo games,
91\% das vezes os país estão presentes na hora que um jogo é alugado ou comprado. 68\% dos pais acreditam que jogos fornecem estimulação mental ou educacional,
57\% acreditam que os jogos auxiliam a família a passar mais tempo junta e 54\% acreditam que ajudam as crianças a se juntarem com amigos. 59\% acreditam que esses
jogos proporcionam mais atividade física hoje do que 5 anos atrás.45\% dos pais jogam com seus filhos no mínimo semanalmente, um aumento de 36\% em 2007.
A idade média do jogador é 37 anos, 82\% tem idade acima de 18 anos, 29\% dos jogadores tem idade acima de 50 anos, um aumento de 9\% em relação a 1999.
Um cenário que é certo de aumentar nos próximos anos com lares e centros de idosos em todo o país incorporando vídeo game em suas atividades.
65\% jogam com outros jogadores pessoalmente. 33\% deles disseram que jogar por computador ou vídeo game é a atividade de entretenimento favorita deles.
Os jogadores adultos já jogam a uma média de 12 anos. Os consumidores gastaram um total de mais de 25 bilhões com a indústria de jogos em 2010.
Todo esse cenário nos revela como existe a pratica regular de se jogar tanto por crianças quanto por adultos, assim como o retorno financeiro envolvido nessa
indústria. É importante então observar e se estudar os efeitos dessa rotina.

 Esse trabalho apresenta uma introdução aos fundamentos teóricos do curso de graduação ciência da computação aplicados ao desenvolvimento de jogos eletrônicos
fisicamente interativos.Ao final, há o detalhamento do processo de desenvolvimento de um jogo demonstrativo, que acompanha a monografia.

\section{Grandes empresas e jogos fisicamente interativos}

 Quando o Nintendo Wii começou a usar controles simples de se usar com sensores de movimentos e com preço mais acessível, criando um ramo de jogos onde sensação
de imersão se dava pelos movimentos físicos que o jogador fazia ao invés de gráficos superiores, fez um grande sucesso, atraiu um grupo de consumidores diferente
e conseguiu competir em vendas contra os consoles  da concorrência que possuem hardware superior, porem alguns anos depois a Microsoft e Sony observando o sucesso
do estilo de jogo, construíram acessórios para desenvolverem também seu jogos fisicamente interativos.

 A Sony desenvolveu um controle similar ao da Nintendo, chamado PS Move, o aparelho possui sensores com desempenho melhores, funciona junto com uma webcam que
possui uma taxa de captura de quadros acima da média, o controle possui uma esfera na extremidade superior que acende com uma luz que varia de cor, essa luz
permite que a webcam identifique o controle mesmo em ambiente completamente escuros, o fato de ter uma webcam trabalhando em conjunto com os controles, melhora
a imersão e faz com que seja mais difícil o usuário "enganar" o jogo, como acontecia por exemplo no Nintendo Wii, era possível jogar um jogo de luta e dar simular
os socos simplesmente balançando os controles, a webcam, no caso da Sony, identifica quando o controle esta mais perto ou mais distante da webcam, colocada em
cima da televisão,balançar os sensores do controle nessa situação não levará o avatar a socar, pois a webcam não estará capturando a aproximação dos controles em
relação a webcam.(colocar referencia motion gaming review)

 A Microsoft criou um acessório diferenciado, com preço bem acima dos concorrentes, eliminou a necessidade de qualquer controle para jogar, o hardware desse
acessório contém 2 câmeras, 4 microfones, acelerômetro, ventoinha, projetor infra-vermelho e diversos chips para que o processamento fosse feito o mínimo no
console.\cite{InsideKinect}

 Apesar da Microsoft fornecer os drivers para que o acessório possa ser reconhecido pelo sistema operacional Windows e mais recentemente o GNU/Linux, o preço está
em torno de R\$ 400,00, o que a torna um acessório inviável para muitos usuários de computador,  contra, por exemplo R\$ 80,00 da PS EYE, nome da webcam
utilizada pelo PS3, preço melhor quando comparada a outras webcams da mesma categoria.

\section{Objetivos do Projeto}

 Desenvolver alguma técnica para capturar os movimentos do usuário através de uma webcam que depois de processado e interpretado irá ser associado a um elemento
que o representará no ambiente virtual.

\subsection{Objetivo geral}

 Estabelecer um esquema modular e criar uma ferramenta para incentivar o desenvolvimento e pesquisas de tecnologias de software para jogos fisicamente interativos
baseados em princípios de imersão.

\subsection{Objetivo específico}

 Implementar um ou mais exemplos de mini-jogos fisicamente interativos, investigando os requisitos mínimos de hardware necessários, como a taxa mínima de quadros
capturados pela webcam, \textit{clock} do processador e memória RAM.

