\chapter{Introdução}

\section{Contextualização e Motivação}

O entretenimento digital vem sendo cada vez mais popular entre as crianças e os adolescentes e atualmente é tão abrangente que engloba um público mais adulto e maduro
e um outro grupo de jogadores casuais que antigamente não jogavam por geralmente encontrar dificuldades em manipular o joystick usual,que agora jogam videogames como
o Nintendo Wii, o PS Move no Playstation 3 e o Kinect no Xbox360. Muitos fabricantes de jogos eletrônicos, se empenham em criar efeitos mais atrativos para os
usuários, uns recorrem a gráficos e animações mais realistas enquanto outros se preocupam em criar  uma experiência de jogabilidade inovadora.\cite{WiiWillRockYou}

Nos jogos fisicamente interativos\footnote{\textit{What is motion-controlled gaming? Definition from WhatIs.com.}
http://whatis.techtarget.com/definition/motion-gaming.html. Acesso  em: 14/09/2011.}, o usuário movimenta partes do corpo para atuar direta ou indiretamente com
o jogo. Esses movimentos são geralmente capturados por controles com sensores de movimento na maioria das vezes com uma câmera, que através de processamento de
imagens, serão traduzidos e retornados como comandos no ambiente virtual do jogo. Essa interação pode ser através de um objeto animado que representa o jogador,
chamado de avatar realizando ações que o usuário realiza na área de visão da câmera.

Atualmente a indústria de jogos tem um movimento financeiro comparável a indústria cinematográfica e um retorno lucrativo muitas vezes maior.
Ela cresceu muito e enquanto antigamente era comum apenas um programador desenvolver um jogo, hoje é necessário quase sempre, centenas de pessoas e milhares de
dólares e alguns anos para a conclusão de um projeto. No fim do ano passado \textit{Call of Duty: Black Ops}\footnote{\textit{Call of Duty: Black Ops} é um jogo de
tiro em primeira pessoa desenvolvido pela Treyarch, editado pela Activision e lançado mundialmente em 9 de novembro de 2010 para as plataformas Microsoft Windows,
 Xbox 360, Playstation 3, Wii e Nintendo DS.} foi o maior lançamento da história do entretenimento,
vendendo no dia do lançamento 5,6 milhões de cópias nos Estados Unidos e Reino Unido, já vendeu mais de 13 milhões e é o título mais vendido da história no mercado
americano. Apenas para efeito de comparação, o filme \textit{Avatar}\footnote{\textit{Avatar} é um filme americano de ficção científica de 2009, escrito e dirigido
por James Cameron, que está em primeiro lugar no ranking de bilheteria, que rendeu US\$ 2,78 bilhões.} vendeu 3,2 milhões de DVDs e Blu-Rays mundialmente no dia do lançamento.\cite{TamanhoIndustriaGames}

Entretenimento por computador ou videogames fazem um papel importante na vida de muitas pessoas e os pais acreditam que existe um impacto positivo em jogar por videogames.
De acordo com a Associação de Software de Entretenimentos do Estados Unidos\cite{ESA}, 72\% dos chefes de família jogam por computador ou vídeo games,
91\% das vezes os país estão presentes na hora que um jogo é alugado ou comprado. 68\% dos pais acreditam que jogos fornecem estimulação mental ou educacional,
57\% acreditam que os jogos auxiliam a família a passar mais tempo junta e 54\% acreditam que ajudam as crianças a se juntarem com amigos. 59\% acreditam que esses
jogos proporcionam mais atividade física hoje do que 5 anos atrás.45\% dos pais jogam com seus filhos no mínimo semanalmente, um aumento de 36\% em 2007.
A idade média do jogador é 37 anos, 82\% tem idade acima de 18 anos, 29\% dos jogadores tem idade acima de 50 anos, um aumento de 9\% em relação a 1999.
Um cenário que é certo de aumentar nos próximos anos com lares e centros de idosos em todo o país incorporando vídeo game em suas atividades.
65\% jogam com outros jogadores pessoalmente. 33\% deles disseram que jogar por computador ou vídeo game é a atividade de entretenimento favorita deles.
Os jogadores adultos já jogam a uma média de 12 anos. Os consumidores gastaram um total de mais de 25 bilhões com a indústria de jogos em 2010.
Todo esse cenário nos revela como existe a pratica regular de se jogar tanto por crianças quanto por adultos, assim como o retorno financeiro envolvido nessa
indústria. É importante então observar e se estudar os efeitos dessa rotina.



Esse trabalho apresenta uma introdução aos fundamentos teóricos do curso de graduação ciência da computação aplicados ao desenvolvimento de jogos eletrônicos
fisicamente interativos

\section{Objetivos do Projeto}

Desenvolver alguma técnica para capturar os movimentos do usuário através de uma webcam que depois de processado e interpretado irá ser associado a um elemento
que o representará no ambiente virtual.

\subsection{Objetivo geral}

Estabelecer um esquema modular e criar uma ferramenta para incentivar o desenvolvimento e pesquisas de tecnologias de software para jogos fisicamente interativos
baseados em princípios de imersão.

\subsection{Objetivo específico}

Implementar um ou mais exemplos de mini-jogos fisicamente interativos, investigando os requisitos mínimos de hardware necessários, como a taxa mínima de quadros
capturados pela webcam, \textit{clock} do processador e memória RAM.

