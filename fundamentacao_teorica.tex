\chapter{Fundamentação Teórica}

\section{Jogos Interativos por Computador}

Alguns dos efeitos negativos dos jogos por computador podem ser resolvidos
incorporando nos jogos interações como movimentos físicos como mecanismo de
controle, em um caso mais promissor, o principio de imersão no jogo , de forma que
 o usuário possa interagir, como se ele mesmo estivesse dentro, com os outros
 elementos.Para isso será necessário módulo de captura de movimento da imagem do
 usuário, detecção e sequenciamento de parte do corpo que transmite a informação
 de controle, análise de informação, tradução para transformar em controle e
 ativação de algum elemento no jogo.

\subsection{Jogos por computador}

Um jogo é uma atividade composta por uma série de ações e decisões, limitado por
regras e por um ambiente virtual, que resultam em um condição final.
Essas regras e ambiente são controlados por um programa digital e existem para
proporcionar uma estrutura e um contexto para as ações do jogador, as regras
não devem ser ambíguas e tem como objetivo criar desafios e se contrapor ao
jogador.Em todo instante existe situações opcionais e negociação por
alternativas, as diferentes condições finais são atribuídos diferentes valores,
o jogador executa as ações de forma a influenciar a condição final.\cite{DesignGames}

\subsection{Interatividade e Imersividade em jogos}

\section{Trabalhos relacionados}

