\chapter{Desenvolvimento}

\section{Introdução}

Foi escolhida a câmera PSeye durante a implementação, devido a seu preço em relação ao que ofere, resolução de
640 x 480 e 320 X 240 pixels, com taxa de quadros de no máximo 60 e 120 frames respectivamente. Assim será possível
avaliar a quantidade ideal de quadros por segundo para gerar uma animação suave sem ter que utilizar outras camera.

\section{Processamento das imagens}

\subsection{OpenCV}

OpenCV é uma biblioteca de visão computacional, escrita em C e C++ e roda em
Linux, Windows e Mac OS X. Com interface para outras linguagens como Python, Ruby, Matlab,
e outras. OpenCV foi projetada para eficiência e aplicativos em tempo real, usar vantagens
de processadores com vários núcleos.
A biblioteca contem mais de 500 funções que se expalham por várias áreas de visão comutacional,
além de módulos de processamendo de imagem e vídeo I/O, estrutura de dados, álgebra linear,
GUI(Inteface Gráfica do Usuário) básica com sistema de janelas independentes, controle de
mouse e teclado, calibração de câmera, reconhecimento de objetos e análise estrutural e
é um software aberto ao uso acadêmico e comercial.\cite{LearningOpenCV}


\section{Construção do ambiente Virtual}

\subsection{OpenGL}

OpenGL(Open Graphics Library) é uma API(Application Programming Interface) gráfica utilizada
na computação gráfica de aplicativos, ambientes virtuais 3D, jogos 2D ou 3D, entre outras várias
aplicações. Possui diversas funções que disponibilizam acesso aos recursos do hardware de vídeo,
facilitando a geração de gráficos 2D e 3D por meio de uma biblioteca uniforme.

\subsection{Pygame}

\subsection{Blender}

Blender é um software multiplataforma de modelagem, animação, texturização, composição, renderização,
edição de vídeo e criação de aplicações interativas em 3D, como jogos, através de seu
motor de jogo integrado, o Blender Game Engine, inclui suporte a Python como linguagem
script, que pode ser usada tanto no Blender quanto em seu motor de jogo e possui código aberto.
Ele está sob a licença GNU-GPL, que permite a qualquer pessoa ter acesso ao código-fonte do programa.\cite{Blender3D}

