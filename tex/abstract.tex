\begin{abstract}

The use of gestures in human-computer interaction provides a powerful justification for research in modeling, analysis and recognition of gestures. In recent years, many new ways to interact with the computer were created, among them, involving computer vision, pattern recognition and artificial intelligence. We have developed an interaction mode with an agent as an avatar in a virtual environment through continuous recognition of predetermined gestures in front of a static background.

The system consists of five modules: detection and extraction of the silhouette of the user using images taken from a camera in real time, extraction and creation of feature vectors, the training of hidden Markov models (HMM), identification of gestures and the avatar actions in virtual environment.

First, background extraction was used to extract the region containing the silhouette of the user, then mesh features is used to feed the feature vector, vector quantization by k-means on these vectors to create the symbols that will be used in the training process and classification of hidden Markov chains. Each sequence of symbols is evaluated in all models and the one that return the greatest probability indicates the gesture to be executed by the avatar in the virtual environment.

\textbf {Keywords: } Hidden Markov Models, Virtual Environments, Human-Computer Interaction (HCI).

\end{abstract}

